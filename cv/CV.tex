\documentclass[a4paper]{twentysecondcv} % a4paper for A4

\newcommand\skills{
        {\begin{itemize}
             \item thorough understanding of algorithms and data structures
             \item working with Git, SVN, GitLab, GitHub, Jenkins, TeamCity
             \item unit and integration testing
             \item knowledge of design patterns
             \item concurrent programming techniques
             \item build process automation with docker and bash/python
    \end{itemize}}
}

\programming{{Rust $\textbullet$ C++ $\textbullet$ C $\textbullet$ Python / 3}, {TypeScript $\textbullet$ HTML $\textbullet$ CSS$\textbullet$ SQL / 4}, {Haskell $\textbullet$ Scala / 4.5}, {Java $\textbullet$ Kotlin/ 5.5}}

\education{

    \textbf{MSc, Computer Science} \\
    University of Warsaw \\
    2021 - Present | Warsaw, Poland

    \textbf{BSc, Computer Science} \\
    University of Warsaw \\
    2018 - 2021 | Warsaw, Poland \\
    \href{https://procyk.in/uploads/diploma.jpg}{GPA: 4.68/5.00}, \href{https://procyk.in/uploads/thesis.pdf}{Bachelor’s thesis}
}

\otherthings {
        {\begin{itemize}
             \item love swimming, running, calisthenics, climbing, cycling and \href{https://procyk.in/uploads/skate.mp4}{skating}
             \item like code reviews and engaging discussions after code debugging
             \item try to be flexible and creative problem solver
    \end{itemize}}
}

%----------------------------------------------------------------------------------------
%	 PERSONAL INFORMATION
%----------------------------------------------------------------------------------------
% If you don't need one or more of the below, just remove the content leaving the command, e.g. \cvnumberphone{}

\cvname{MACIEJ PROCYK} % Your name
\cvjobtitle{} % Job
% title/career

\cvlinkedin{/in/maciej-procyk}
\cvgithub{/avan1235}
\cvgitlab{/avan1235}
\cvnumberphone{+48 723 488 306} % Phone number
\cvsite{https://procyk.in} % Personal website
\cvmail{maciej@procyk.in} % Email address

%----------------------------------------------------------------------------------------

\begin{document}
    \makeprofile % Print the sidebar

    \vspace{-0.3cm}


    \section{Experience}
    \begin{twenty}
        \twentyitem
        {Jul. 2022}
        {Sep. 2022}
        {AppCode Internship in JetBrains}
        {}
        {}
        {\begin{itemize}
             \item working on the support of Swift language in AppCode IDE (focused on inline refactorings for functions and variables)
             \item using Kotlin, Intellij SDK, Gradle, Space, YouTrack, TeamCity
        \end{itemize}
        }
        \\
        \twentyitem
        {Mar. 2022}
        {Jun. 2022}
        {Teaching Assistant at University of Warsaw}
        {}
        {}
        {\begin{itemize}
             \item group instructor for Object-Oriented Programming with Java - classes and laboratory
        \end{itemize}
        }
        \\
        \twentyitem
        {Jul. 2021}
        {Sep. 2021}
        {AppCode Internship in JetBrains}
        {}
        {}
        {\begin{itemize}
             \item working on the features of Kotlin Mutliplatform Mobile plugin
             \item using Kotlin, Intellij SDK, Gradle, Space, YouTrack, TeamCity
        \end{itemize}
        }
        \\
        \twentyitem
        {Aug. 2020}
        {Mar. 2021}
        {Junior Java Developer in SoftwarePlant}
        {}
        {}
        {\begin{itemize}
             \item working on Java Spring application for projects management
             \item using Guava, Hibernate, Maven, JUnit, Wiremock, PostgreSQL, Docker
        \end{itemize}
        }
        \\
        \twentyitem
        {Jul. 2019}
        {Oct. 2019}
        {Java EE Internship in Accenture}
        {}
        {}
        {\begin{itemize}
             \item working on Java EE web application for problems management
             \item using Maven, Hibernate, GWT, SQL
        \end{itemize}
        }
        \\

    \end{twenty}

    \vspace{-0.6cm}

    \section{Projects}

    \begin{twenty}
      \twentyitem
      {Dec. 2021}
      {Mar. 2023}
      {\href{https://github.com/avan1235/mini-games}{Mini Games {\footnotesize @ GitHub}}}
      {}
      {}
      {\begin{itemize}
         \item multiplatform app for every popular platform, including Desktop, Android, iOS
         \item designed server responsible for games' physics and clients presenting the current state to users using fully shared Kotlin code for model as well as for all clients
         \item using Kotlin Multiplatform, Ktor, Exposed, Compose Multiplatform, Kotlin Serialization and Coroutines
      \end{itemize}}
      \\
        \twentyitem
        {Oct. 2021}
        {Jan. 2022}
        {\href{https://github.com/avan1235/latte-compiler}{Latte Native compiler {\footnotesize @ GitHub}}}
        {}
        {}
        {\begin{itemize}
             \item implemented part of Latte language compiler fox x86 supporting basic types, loops, classes,
             virtual class methods, register allocation and code optimisations
             \item using ANTLR, JUnit, GraalVM, Kotlin
        \end{itemize}}
        \\
        \twentyitem
        {Feb. 2021}
        {May. 2021}
        {\href{https://gitlab.com/avan1235/kotlin-interpreter}{Kotlin interpreter {\footnotesize @ GitLab}}}
        {}
        {}
        {\begin{itemize}
             \item implemented part of Kotlin language supporting basic types, arrays, loops, functions
             (with any types as return types and arguments), final variables,
             handling runtime errors, higher-order functions and sequences
             \item using BNFC, stack, happy, alex, make, Haskell
        \end{itemize}}
        \\
        \twentyitem
        {Nov. 2020}
        {May. 2021}
        {\href{https://github.com/prinz-nussknacker/prinz}{Prinz {\footnotesize @ GitHub}}}
        {}
        {}
        {\begin{itemize}
             \item provide machine learning integrations for \href{https://github.com/TouK/nussknacker}{Nussknacker}
             \item BSc thesis project designed in 4-person team coordinated by external supervisor
             \item using MLflow, jpmml, H2Oai, Docker, Scala
        \end{itemize}}
        \\
        \twentyitem
        {Feb. 2020}
        {Jun. 2020}
        {\href{https://gitlab.com/avan1235/remotebot}{RemoteBot {\footnotesize @ GitLab}}}
        {}
        {}
        {\begin{itemize}
             \item system for WiFi controlling LEGO Mindstorms NXT 2.0 robot with video stream
             \item developed server and Android client app for robot controlling
             \item using leJOS (Java robot programming), flask (OrangePi server) and Kotlin (Android app client)
        \end{itemize}}
        \\
    \end{twenty}

    \vspace{-0.6cm}

    \section{Scholarships and Awards}
    \begin{twenty}
        \twentyitem
        {2022/23}
        {}
        {winner of Kotlin Multiplatform Contest}
        {Kotlin Foundation}{\href{https://github.com/avan1235/mini-games}{Mini Games {\footnotesize @ GitHub}} announced on \href{https://twitter.com/kotlinconf/status/1616430150836445184}{Twitter}}{}
        \twentyitem
        {2019/20}
        {}
        {winner of Mathematica student project}
        {University of Warsaw}{Research on Cellular Automata in Wolfram Language}{}
        \twentyitem
        {2017/18}
        {}
        {winner (2018) and laureate (2017) of Polish Olympiad of Technical Knowledge, finalist of Polish Olympiad of Physics}
        {Warsaw}{}{}
        \twentyitem
        {2017}
        {}
        {3rd place in European CanSat Competition}
        {Bremen}{Build minisatellite as a member of CANpernicus team}{}
    \end{twenty}


%  \vspace{-0.4cm}
%
%  \section{Research}
%  P. Hanczyc, M. Procyk, C. Radzewicz, P. Fita - Steady-state and Time-Resolved Spectroscopic Characterization of Coumarin 307: Its
%  Aggregation and Two-Photon Excited Stimulated Emission for Lysozyme Amyloid Fibrils Detection

\end{document}
